\documentclass[acmtog]{acmart}

\def\BibTeX{{\rm B\kern-.05em{\sc i\kern-.025em 
b}\kern-.08emT\kern-.1667em\lower.7ex\hbox{E}\kern-.125emX}}

\begin{document}

\title{Towards Detecting Stalkerware Through Network Traces}


\author{Audrey Randall}
\author{Hugh Feng}
\author{Yu-Wun Wang}
\affiliation{\institution{UCSD}}

\begin{abstract}
Spyware is known as unwanted software that sends user state and information to 
an external party. Often these apps are not inherently malicious to the 
operating system and are thus difficult to detect. We explore a subset of 
spyware that are available in the Android Play store intended for spying on 
another person though his or her cell phone and create a new method of 
identifying their presence through network signatures.  
\end{abstract}

\maketitle

\section{Introduction}
In a world where technology now pervades every aspect of our lives, some 
aspects of our relationships with intimate partners have changed too. 
Specifically, when these relationships become abusive, there are more options 
available to the abuser for attempting to retain control over their partner. 
One of these options is the ability to install monitoring software on their 
partner's phone or computer to record their online activity, communications, 
and location. We refer to such software as ``stalkerware.''  In a survey of 72 
domestic violence shelters performed by NPR in 2014, 85\% said they worked with 
survivors whose abusers had tracked their location using their smartphone's 
GPS. 75\% said they are working with clients who have had their conversations 
monitored by hidden stalkerware apps.\cite{shahani_smartphones}

Although stalkerware is superficially similar to other types of malware, a few 
key differences exist between them. First, stalkerware is meant to be installed 
deliberately, by someone with physical access to the device that is to be 
monitored. In the context of intimate partner abuse, this presents an enormous 
threat. An abuser will often know or be able to coerce passwords to the device 
from the victim, making traditional defense measures irrelevant. Second, the 
threat model used by malware developers includes sophisticated antivirus and 
intrusion detection systems (IDSs). Malware is therefore much more likely to 
attempt to obfuscate its network traffic. Third, stalkerware falls for the most 
part into two categories: apps that are explicitly advertised for surveilling 
spouses and partners, and apps that have a different stated use but can be 
repurposed to stalk a target. These are usually marketed for theft prevention, 
parental control of children, and employer control of employees. Previous 
literature refers to these as "dual-use" apps [5]. Determining how many users 
of a particular dual-use app are using it for intimate partner surveillance 
(IPS), and how many for its stated (usually legal) purpose, presents a 
significant challenge.

Therefore, we start by identifying a set of dual-use apps that appear to be 
marketed primarily for IPS. Apps that appear in Google searches for terms like 
"catch a cheating spouse," use vague language that appear to be marketed 
towards IPS, and label themselves as "spyware" became candidates that we 
considered. Such apps also differentiated themselves from more innocent apps by 
their price - they cost much more than a typical app, with monthly 
subscriptions ranging from \$12 to \$68 per month. We selected several of these 
to test. All of these apps brand themselves as spyware and advertise that their 
products are undetectable on the target phone, despite their disclaimers that 
they should not be used for any illegal activity. Some use language that 
appears designed to appeal to abusers. For example, underneath one promotional 
video, the app TheOneSpy has the disclaimer: "Note for this video: Don't 
confuse with the world "Loved Ones" as spouse or Girl friends! Word Loved ones 
has used for kids and teens." Many of these apps also refer to the tracked 
phone as the "target" phone, and advertise that once installed, the app is in 
"stealth mode," meaning completely undetectable. 

The other category for stalkerware encompasses a large range of quality. Many 
apps advertised as stalkerware have absolutely no functionality other than 
advertisement of other apps. The few that are functional are often simplistic 
in both features and network patterns. Unlike dual-use apps, however, the vast 
majority are usually free with advertisements. These apps can be easily 
identified and found with a few keywords such as "spy","track", "locate", or 
"find". 

\section{Related Work}
Previous studies [1,2], the media [3,4], and intimate partner violence (IPV) 
survivors have all reported that spyware is becoming an increasing source of 
intimate partner surveillance (IPS). Tracking functions supported by spyware 
allow stalkers to monitor communications, track locations, and remotely 
activate cameras and microphones [3,4]. In Dr. Ristenpart’s work [5], they 
proposed methods for searching IPS-related apps. By Google search query 
expansion and unsupervised classification, they collected 9,224 apps possibly 
related to IPS on Google play and open-web. They manually selected 70 apps for 
investigation. They observed the most fundamental functions of these apps are 
monitoring locations, communication logs and data, media contents, and phone 
usages. They also found that existing anti-spyware tools cannot effectively 
detect IPS-related apps.
Mobile apps are often identified via the User-Agent string of the HTTP request, 
when one is available [7]. With the increased use of HTTPS this may not always 
be the case, but we expect to find (based on conversations with authors of 
previous work) that security and best coding practices are not prioritized by 
stalkerware developers. Malware identification may be of limited use as well, 
since techniques are available to cluster malware that comes from the same 
source code into “families.” [8]

\section{Methodology}
\subsection{Finding Apps}
We selected thirty apps from a list provided by the team of Cornell researchers 
who performed the original large scale study of stalkerware[cite]. An 
additional fifteen apps were included as they were recently added to the app 
store. Apps were chosen for their diversity of purpose and the probability that 
they can be used as stalkerware. These apps are the most likely to be 
downloaded when a abuser without deep technical knowledge tries to install 
stalkerware. 

We recognize that there are thousands of apps on the market that can be 
repurposed as stalkerware. Therefore, any useful process for identifying 
network signatures will eventually have to cover far more apps than we are 
starting out with. Our mere forty-five apps barely scratch the surface of the 
ecosystem, so they certainly do not provide a representative sample. Therefore, 
we did not employ any vigorous app selection methods, since we know this is 
just a starting point. Our hope is to eventually find a way to partially 
automate the signature identification process.

However, some characteristics of the stalkerware apps themselves may aid us in 
detecting them. For example, similar to malware, some groups of apps appear to 
be repackaged versions of each other. For example, PhoneSpector, 
HighsterMobile, AutoForward, EasySpy and Surepoint Spy all have separate 
websites, but the appearance of each of these sites is nearly identical. Upon 
closer inspection, although multiple companies are listed as the makers of the 
apps (Powerline Group Corp., ILF Mobile Apps, and PhoneSpector LLC), all have 
the same physical address listed. At least one blog that is supposedly run by 
an independent part reviewed apps that coincidentally were made by the same 
company [9]. Other app sites appear to be similar to each other in other ways, 
with many having a large panel with a darkened image background extolling the 
features of the app above two buttons, one for a “live demo” and the other to 
“buy now.” This is somewhat circumstantial evidence - it is entirely possible 
that these manufacturers simply chose similar web templates, or copy each 
other’s site in an attempt to undercut competition. However, if multiple groups 
of stalkerware apps are in fact made by only a few manufacturers, it may 
facilitate app identification. Similarities in network traces may indicate that 
the code used in the apps was made by the same developers.

\subsection{App Traffic}

Upon identifying the apps and locating them on the app store or app website, we 
proceeded to install the apps on two phones. One is a (what are our phones). 
After installing and confirming functionality, we began tracking the phone's 
network traffic using Wireshark. For the majority of the apps, a network trace 
of five to ten minutes is enough to detect traffic going to and from the 
stalkerware app as long as the app's tracked functions like text or GPS are 
altered or the tracker app or website has a manual update function. Those that 
do not have such functionality have all been fake stalkerware apps. The network 
captures themselves contain at least one packet that provides easy manual 
detection of the related app.

In addition to the above network captures, we also took twenty-four hours of 
network traffic of our own phones as a baseline for false positive rates. These 
captures also included periods of time that were deemed high risk for false 
positives as they emulated behaviour that is most likely to be caught by our 
signatures. Although this may not be enough to fully emulate an entire network, 
it is enough to prove the false positive rates on any given phone.

\section{Traffic Results}

The majority of stalkerware apps communicate though HTTP or HTTPS.

\section{temp}

Some examples.  A paginated journal article \cite{Abril07}, an enumerated 
journal article \cite{Cohen07}, a reference to an entire issue \cite{JCohen96}, 
a monograph (whole book) \cite{Kosiur01}, a monograph/whole book in a series 
(see 2a in spec. document)
\cite{Harel79}, a divisible-book such as an anthology or compilation 
\cite{Editor00} followed by the same example, however we only output the series 
if the volume number is given \cite{Editor00a} (so Editor00a's series should 
NOT be present since it has no vol. no.),
a chapter in a divisible book \cite{Spector90}, a chapter in a divisible book 
in a series \cite{Douglass98}, a multi-volume work as book \cite{Knuth97}, an 
article in a proceedings (of a conference, symposium, workshop for example) 
(paginated proceedings article) \cite{Andler79}, a proceedings article with all 
possible elements \cite{Smith10}, an example of an enumerated proceedings 
article \cite{VanGundy07}, an informally published work \cite{Harel78}, a 
doctoral dissertation \cite{Clarkson85}, a master's thesis: \cite{anisi03}, an 
online document / world wide web resource \cite{Thornburg01, Ablamowicz07, 
Poker06}, a video game (Case 1) \cite{Obama08} and (Case 2) \cite{Novak03} and 
\cite{Lee05} and (Case 3) a patent \cite{JoeScientist001}, work accepted for 
publication \cite{rous08}, 'YYYYb'-test for prolific author \cite{SaeediMEJ10} 
and \cite{SaeediJETC10}. Other cites might contain 'duplicate' DOI and URLs 
(some SIAM articles) \cite{Kirschmer:2010:AEI:1958016.1958018}. Boris / Barbara 
Beeton: multi-volume works as books \cite{MR781536} and \cite{MR781537}. A 
couple of citations with DOIs: 
\cite{2004:ITE:1009386.1010128,Kirschmer:2010:AEI:1958016.1958018}. Online 
citations: \cite{TUGInstmem, Thornburg01, CTANacmart}.



%
% The next two lines define the bibliography style to be used, and the 
%bibliography file.
\bibliographystyle{ACM-Reference-Format}
\bibliography{IPS_Measurement.bib}

% 
% If your work has an appendix, this is the place to put it.
\appendix

\section{Apps and Signatures}



\end{document}
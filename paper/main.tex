\documentclass[acmtog]{acmart}

\def\BibTeX{{\rm B\kern-.05em{\sc i\kern-.025em 
b}\kern-.08emT\kern-.1667em\lower.7ex\hbox{E}\kern-.125emX}}

\begin{document}

\title{Towards Detecting Stalkerware Through Network Traces}


\author{Audrey Randall}
\author{Hugh Feng}
\author{Yu-Wun Wang}
\affiliation{\institution{UCSD}}

\begin{abstract}
Spyware is known as unwanted software that sends user state and information to 
an external party. Often these apps are not inherently malicious to the 
operating system and are thus difficult to detect. We explore a subset of 
spyware that are available in the Android Play store intended for spying on 
another person though his or her cell phone and create a new method of 
identifying their presence through network signatures.  
\end{abstract}

\maketitle

\section{Introduction}
In a world where technology now pervades every aspect of our lives, some 
aspects of our relationships with intimate partners have changed too. 
Specifically, when these relationships become abusive, there are more options 
available to the abuser for attempting to retain control over their partner. 
One of these options is the ability to install monitoring software on their 
partner's phone or computer to record their online activity, communications, 
and location. We refer to such software as ``stalkerware.''  

Although stalkerware is superficially similar to other types of malware, it has 
a few key differences. First, stalkerware is meant to be installed 
deliberately, by someone with physical access to the device that is to be 
monitored. In the context of intimate partner abuse, this presents an enormous 
threat. An abuser will often know or be able to coerce passwords to the device 
from the victim, making traditional defense measures irrelevant. Second, the 
threat model used by malware developers includes sophisticated antivirus and 
intrusion detection systems (IDSs). Malware is therefore much more likely to 
attempt to obfuscate its network traffic. Third, stalkerware falls for the most 
part into two categories: apps that are explicitly advertised for surveilling 
spouses and partners, and apps that have a different stated use but can be 
repurposed to stalk a target. These are usually marketed for theft prevention, 
parental control of children, and employer control of employees. Previous 
literature refers to these as "dual-use" apps\cite{chatterjee_spyware_2018}. 
Some dual-use apps are ones that the survivor wants to keep installed, such as 
``find my phone'' apps or social media apps that leak the user's location. The 
most troubling example of this is Facebook, which most survivors want to keep 
so that they can contact their support network for help.  
Determining if a particular dual-use app is being used for 
intimate partner surveillance (IPS) presents an enormous challenge. Given the 
wide spectrum of apps that can be used to facilitate IPS and the limited scope 
of this project, we chose to focus on more overt spyware.

Largely due to the differences between stalkerware and other malware, most 
available information on the pervasiveness of stalkerware is anecdotal and 
piecemeal. Small-scale studies on how widespread spyware is do exist, but they 
have limitations. For example, in a survey of 72 
domestic violence shelters performed by NPR in 2014, 85\% said they worked with 
survivors whose abusers had tracked their location using their smartphone's 
GPS. 75\% said they are working with clients who have had their conversations 
monitored by hidden stalkerware apps.\cite{shahani_smartphones_nodate} However, 
this study does not report how many individual clients experienced being 
digitally stalked. Additionally, it and similar work relies on 
reports from domestic violence shelters and health workers, which does not give 
a complete picture of the situation. Little to no data exist regarding spyware 
use on IPA victims who never visit a shelter. Traditionally, antivirus vendors 
collect metrics about 
malicious software, but antivirus often does not detect spyware 
\cite{chatterjee_spyware_2018}. Alarmingly, the stalkerware 
websites themselves claim to have hundreds of thousands of downloads per app, 
but these numbers are not likely to be trustworthy. The goal of this work is 
therefore to identify a means of identifying stalkerware in the wild so that it 
can be enumerated. We collected the network signatures of 24 stalkerware apps 
and showed that they could be easily distinguished from ordinary smartphone 
network traffic. We also propose a simple method of generalizing our signatures 
to spyware that we did not manually examine, and evaluate its robustness.


We began by identifying a set of apps that appear to be 
marketed primarily for IPS. We focused exclusively on Android apps, since our 
two test phones were both Androids. Apps that appear in Google searches for 
terms like 
"catch a cheating spouse," use vague language that appear to be marketed 
towards IPS, and label themselves as "spyware" became candidates that we 
considered. Overt spyware is not generally available from the app store - it 
must be downloaded from the manufacturer sites. It also costs much more than a 
typical app, with monthly 
subscriptions ranging from \$12 to \$70 per month. The exception is an app that 
we call SpyToMobile, which survives on the app store by frequently 
``reskinning'' itself (see section \ref{reskinning}).  
We selected several of these to test. All of these apps brand themselves as 
spyware and advertise that their products are undetectable on the target phone, 
despite also having disclaimers in small text that their software should not be 
used for any illegal activity. Some use language that appears designed to 
appeal to abusers. For example, underneath one promotional 
video, the app TheOneSpy has the disclaimer: ``Note for this video: Don't 
confuse with the world `Loved Ones' as spouse or Girl friends! Word Loved ones 
has used for kids and teens.'' Many of these apps also refer to the tracked 
phone as the ``target'' phone, and advertise that once installed, the app is in 
``stealth mode,'' meaning completely undetectable. None of these required 
rooting the phone. When installed, they ask for all the permissions they 
require, including device administrator privileges. They are capable of 
blocking incoming calls, recording messages that have been deleted, recording 
the screen of the phone using accessibility features, monitoring apps such as 
Facebook, WhatsApp, Kik, and Viber, and more.

Other apps were overt spyware, but nevertheless available from the Play Store. 
They fell into two categories: spyware that reskinned itself to appear to be 
many different apps, and apps designed for parents to keep track of children 
(or in one case, for adult children to track elderly parents.) For the most 
part, this latter category was offered by antivirus companies such as Avast and 
Norton. These apps offered 
multiple features, such as call screening, blocking Google search results, 
recording text messages and call logs, and ``panic buttons'' that would allow 
the surveilled phone to contact the surveillant and raise an alarm. They do not 
hide their app icons and generally make an effort to notify the user that the 
monitoring app is installed on their phone. The spyware that did not notify 
users of its presence mostly consisted of one app, called SpyToMobile, that was 
provided on the Play Store under many different names. 

We also identified many apps that were designed to track location or text 
messages, but were less comprehensive than the overt, off-store spyware. They 
track one feature each instead of tracking location, communication, and 
browsing history all at once. For the most part, these apps were free and 
available on the Google Play Store. Most of these were designed to keep track 
of the locations of the users' friends. 

\section{The Stalkerware Ecosystem}
\subsection{The Practice of ``Reskinning''}
\label{reskinning}

Somewhere: put a note about the weird thing where the paid apps try to make you 
download stuff again. Maybe it's just us? Maybe they're trying to boost 
download count?

Some spyware has a notification that the app is running, but that can be 
disabled. Those ones don't hide their icon, but you might have to go to All 
Apps to find it.

\subsection{App Quality}

The other category for stalkerware encompasses a large range of quality. Many 
apps advertised as stalkerware have absolutely no functionality other than 
advertisement of other apps. The few that are functional are often simplistic 
in both features and network patterns. Unlike the overt apps, however, the vast 
majority are usually free with advertisements. These apps can be easily 
identified and found with a few keywords such as "spy","track", "locate", or 
"find". 

\section{Related Work}
Previous studies [1,2], the media [3,4], and intimate partner violence (IPV) 
survivors have all reported that spyware is becoming an increasing source of 
intimate partner surveillance (IPS). Tracking functions supported by spyware 
allow stalkers to monitor communications, track locations, and remotely 
activate cameras and microphones [3,4]. In Dr. Ristenpart’s work [5], they 
proposed methods for searching IPS-related apps. By Google search query 
expansion and unsupervised classification, they collected 9,224 apps possibly 
related to IPS on Google play and open-web. They manually selected 70 apps for 
investigation. They observed the most fundamental functions of these apps are 
monitoring locations, communication logs and data, media contents, and phone 
usages. They also found that existing anti-spyware tools cannot effectively 
detect IPS-related apps.
Mobile apps are often identified via the User-Agent string of the HTTP request, 
when one is available [7]. With the increased use of HTTPS this may not always 
be the case, but we expect to find (based on conversations with authors of 
previous work) that security and best coding practices are not prioritized by 
stalkerware developers. Malware identification may be of limited use as well, 
since techniques are available to cluster malware that comes from the same 
source code into “families.” [8]

\section{Methodology}
\subsection{Finding Apps}
We selected thirty apps from a list provided by the team of Cornell researchers 
who performed the original large scale study of 
stalkerware\cite{chatterjee_spyware_2018}. An 
additional fifteen apps were included as they were recently added to the app 
store. Apps were chosen for their diversity of purpose and the probability that 
they can be used as stalkerware. These apps are the most likely to be 
downloaded when a abuser without deep technical knowledge tries to install 
stalkerware since they are either the first result upon searching for 
stalkerware, or are well known spyware apps due to advertisement or other 
reasons. 

We recognize that there are thousands of apps on the market that can be 
repurposed as stalkerware. Therefore, any useful process for identifying 
network signatures will eventually have to cover far more apps than we are 
starting out with. Our mere forty-five apps barely scratch the surface of the 
ecosystem, so they certainly do not provide a representative sample. Therefore, 
we did not employ any vigorous app selection methods, since we know this is 
just a starting point. Our hope is to eventually find a way to partially 
automate the signature identification process.

However, some characteristics of the stalkerware apps themselves may aid us in 
detecting them. For example, similar to malware, some groups of apps appear to 
be repackaged versions of each other. For example, PhoneSpector, 
HighsterMobile, AutoForward, EasySpy and Surepoint Spy all have separate 
websites, but the appearance of each of these sites is nearly identical. Upon 
closer inspection, although multiple companies are listed as the makers of the 
apps (Powerline Group Corp., ILF Mobile Apps, and PhoneSpector LLC), all have 
the same physical address listed. At least one blog that is supposedly run by 
an independent part reviewed apps that coincidentally were made by the same 
company\cite{noauthor_best_nodate}. Other app sites appear to be similar to each other in other ways, 
with many having a large panel with a darkened image background extolling the 
features of the app above two buttons, one for a "live demo" and the other to 
"buy now." This is somewhat circumstantial evidence - it is entirely possible 
that these manufacturers simply chose similar web templates, or copied each 
other's site in an attempt to undercut competition. However, if multiple groups 
of stalkerware apps are in fact made by only a few manufacturers, it may 
facilitate app identification. 

After our investigation on these apps, we can confirm that, at least from the few that we have selected, many stalkerware apps are related in some fashion to each other. The most notable being the group of apps stated above, and a single app called spy2mobile. Spy2mobile is an interesting case because it fits the description of a generic stalkerware app so closely that it is the perfect example of spyware on the Play Store. Moreover, this single app has been repackaged multiple times. We have discovered that there are at least 6 copies of the app on the Play store, all with different developers and different names. While this is still circumstantial evidence that most stalkerware often contains copies of each other, we can at least conclue that our forty-five apps constitute a larger percentage than anticipated.

\subsection{App Traffic}

Upon identifying the apps and locating them on the app store or app website, we 
proceeded to install the apps on two phones. One is a (what are our phones). 
After installing and confirming functionality, we began tracking the phone's 
network traffic using Wireshark. For the majority of the apps, a network trace 
of five to ten minutes is enough to detect traffic going to and from the 
stalkerware app as long as the app's tracked functions like text or GPS are 
altered or the tracker app or website has a manual update function. Those that 
do not have such functionality have all been fake stalkerware apps. The network 
captures themselves contain at least one packet that provides easy manual 
detection of the related app.

What causes most stalkerware to send packets? Our captures and testing show that there are three main categories of updating tracker state: manual, timed, and on tracked feature. Most free apps fall into the manual update method, where there is a button on a website or on a phone app that will send a signal packet to the stalkerware app that initalizes the upload. Most paid apps also include a timed auto-update feature, where an update will be sent every few minutes. A very small subset of both advertised that it receives a "real time notification" every time email, texts, calls, or GPS, was changed or added to. Our results show that none of the stalkerware apps were able to send in real time.

In addition to the above network captures, we also took twenty-four hours of 
network traffic of our own phones as a baseline for false positive rates. These 
captures also included periods of time that were deemed high risk for false 
positives as they emulated behaviour that is most likely to be caught by our 
signatures. Although this may not be enough to fully emulate an entire network, 
it is enough to prove the false positive rates on any given phone.

\subsection{Network Signatures}


\section{Traffic Results}

The majority of stalkerware apps communicate though HTTP or HTTPS, while some use their own protocols. We examined the domain, SNI, User-Agent, and HTTP host, for each app. The words “spy,” “find,” and “track” appeared frequently in various fields of the traffic. The most common place to find them was in the destination domain, which we found using Wireshark’s reverse DNS lookup tool. 12 of the 26 apps that we eventually found signatures for contained these spyware keywords in their domains. We identified the following generalized signatures that can be used to identify most of our apps given nothing but information available on the Google Play store.



\section{Future Work}

Make machine learning identify stalkerware on the app store, download it, 
profile it. Then train more ML to pick out the malicious signatures? Would that 
even work? What would it learn that we can't?

\section{Conclusion}

\bibliographystyle{ACM-Reference-Format}
\bibliography{IPS_Measurement}

\appendix

\section{Apps and Signatures}

\end{document}
